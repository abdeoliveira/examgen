%% LyX 2.2.3 created this file.  For more info, see http://www.lyx.org/.
%% Do not edit unless you really know what you are doing.
\documentclass[9pt,twocolumn,brazil]{article}
\usepackage[T1]{fontenc}
\usepackage[utf8]{inputenc}
\usepackage{fancyhdr}
\usepackage{graphicx}
\usepackage{amsmath}
\usepackage{geometry}
%%%%% GENEXAMS SPECIFIC %%%%%%%%%%%%
\mathcode`,="002C
\usepackage{tabularx}
%\usepackage[locale=FR]{siunitx}
%\sisetup{per-mode=symbol}
\geometry{verbose,tmargin=1cm,bmargin=2cm,lmargin=1cm,rmargin=1cm}
\usepackage{xcolor,colortbl}
\definecolor{black}{rgb}{0,0,0}
\newcommand{\black}{\cellcolor{black}}  %{0.9}

\makeatletter
%%%%%%%%%%%%%%%%%%%%%%%%%%%%%% User specified LaTeX commands.
\usepackage{setspace}
\usepackage[sort]{cite}
%\pagestyle{empty}
\pagestyle{fancy}
\fancyhead{}
\fancyfoot{}
\renewcommand{\headrulewidth}{0.0pt}
\fancyfoot[RO,RE]{1/2022.12.22-16.36.18}

\makeatother

\usepackage{babel}
\begin{document}

\textbf{Universidade Federal de Ouro Preto}

\textbf{Instituto de Ciências Exatas e Biológicas}

\textbf{Departamento de Física}

\textbf{Prof. Dr. Alan Barros de Oliveira}\\

Prova 2 -- FIS110-73 -- 17/06/2022\medskip{}

%\textbf{Considere a aceleração gravitacional $g=\SI{9.8}{\meter\per\square\second}$}\medskip{}



1. Mark the true alternative.\\
(a) The vectorial product between collinear vectors is zero.\\
(b) Vectors can not be multiplied by scalars.\\
(c) Division between vectors is defined in Mathematics.\\
(d) The result of summing a vector and a scalar is a scalar.\\
(e) The result of a scalar product between vectors is a vector itself.\\

2. A particle of mass 2,7 kg is subject to an external force of 18,9 N. Calculate the acceleration in m/s$^2$ in a one-dimensional movement.\\
(a)17,0~
(b)9,4\\
(c)3,1~
(d)13,0\\
(e)7,0\\

3. Consider the rectangle triangle of the figure below and knowing $\theta=26^{\circ}$, determine $\phi$ \textbf{in rad}.\\
\includegraphics[width=5.0cm]{../input/figs/triangle.png}\\
(a)1,174~
(b)1,059~
(c)1,252~
(d)1,117~
(e)1,305\\


%\noindent\rule{6cm}{0.7pt}\\
{\large \textbf{Fórmulas e Constantes}}
\begin{align*}
%-----------CHAP. 33 and 35-------------
	%&E=E_m\sin(kx-\omega t);~~B=B_m\sin(kx-\omega t)&\\
	%&c=\frac{E}{B}=\frac{E_m}{B_m}={\sqrt{\mu_0 \epsilon_0}};~~\vec{S}=\frac{1}{\mu_0}\vec{E}\times\vec{B}&\\
	%&I=\frac{1}{c\mu_0}\frac{E_m^2}{2};~~I=\frac{P_s}{4\pi r^2};~~P_r=\frac{F}{A}=\gamma\frac{I}{c}~(\mathrm{onde}~1\leq\gamma\leq2)&\\
	%&I=\frac{I_0}{2};~~I=I_0 \cos^2\theta;~~n_1\sin\theta_1=n_2\sin\theta_2~~&\\
	%&\theta_c=\sin^{-1}\frac{n_2}{n_1};~~\theta_B=\tan^{-1}\frac{n_2}{n_1};~~\lambda_n = \frac{\lambda}{n};~~n=\frac{c}{v};~~v=f\lambda&\\
%-----------CHAP. 38 and 39-------------
	&I=\frac{P_s}{4\pi r^2};~~E=hf;~~p=\frac{hf}{c}=\frac{h}{\lambda}&\\
	&hf=K_\mathrm{max}+\Phi;~~\Delta\lambda=\frac{h}{mc}(1-\cos\phi)&\\
	&\frac{d^2\psi}{dx^2}+\frac{8\pi^2 m}{h^2}[E-U(x)]\psi=0&\\
	&T\approx e^{-2bL},~\mathrm{onde}~b=\sqrt{\frac{8\pi^2 m(U_b-E)}{h^2}}&\\
	&E_n=\left(\frac{h^2}{8mL^2}\right)n^2,~\mathrm{para}~n=1,2,3\dots&\\
	&\psi_n(x)=A\sin\left(\frac{n\pi}{L}x\right),~\mathrm{para}~n=1,2,3\dots&\\
	&\Delta x \Delta p = h/2\pi&\\
%--------CONSTANTS----------------------
	&\epsilon_0 = 8,854\times10^{12}~\mathrm{F/m};~~\mu_0 = 1,257\times10^{-6}~\mathrm{H/m}&\\
	&c=3,0\times10^8~\mathrm{m/s};~~h=6,63\times10^{-34}~\mathrm{J/s}=4,14\times10^{-15}~\mathrm{eV.s}&\\
	&hc=1240~\mathrm{eV.nm}&\\
	&\mathrm{Eletron:}~~mc^2 = 511~\mathrm{keV}&\\
%---------------------------------------
\end{align*}
%\rule{6cm}{0.7pt}
\includegraphics[width=6cm]{../aux/matricula.png}\\
\vfill
\pagebreak
\begin{tabular}{!{\vrule width 1.5pt}c|c|c|c|c|c|c|c|c|c|c!{\vrule width 1.5pt}}
\noalign{\hrule height 1.5pt}
	\multicolumn{1}{!{\vrule width 1.5pt}c}{} & \multicolumn{9}{c}{\textbf{NÃO MARCAR}} & \tabularnewline
\hline 
un&--&\black&--&--&--&--&--&--&--&-- \tabularnewline
\hline%un 
%--&--&--&--&--&--&--&--&--&-- \tabularnewline
 
%--&--&--&--&--&--&--&--&--&--  \tabularnewline
 
	\multicolumn{1}{!{\vrule width 1.5pt}c}{} & \multicolumn{9}{c}{\textbf{GABARITO}} &  \tabularnewline
\hline
-- & 1 & 2 & 3 & -- & -- & -- & -- & -- & -- & --\tabularnewline
\hline
a &   &   &   & -- & -- & -- & -- & -- & -- & --\tabularnewline
\hline
b &   &   &   & -- & -- & -- & -- & -- & -- & --\tabularnewline
\hline
c &   &   &   & -- & -- & -- & -- & -- & -- & --\tabularnewline
\hline
d &   &   &   & -- & -- & -- & -- & -- & -- & --\tabularnewline
\hline
e &   &   &   & -- & -- & -- & -- & -- & -- & --\tabularnewline
\hline
\multicolumn{1}{!{\vrule width 1.5pt}c}{} & \multicolumn{9}{c}{\textbf{MATRÍCULA}} & \tabularnewline
\hline 
-- & \phantom{0}0 & \phantom{0}1 & \phantom{0}2 & \phantom{0}3 & \phantom{0}4 & \phantom{0}5 
& \phantom{0}6 & \phantom{0}7 & \phantom{0}8 & \phantom{0}9\tabularnewline
\hline 
1$^{\circ}$ &  &  &  &  &  &  &  &  &  & \tabularnewline
\hline 
2$^{\circ}$ &  &  &  &  &  &  &  &  &  & \tabularnewline
\hline 
3$^{\circ}$ &  &  &  &  &  &  &  &  &  & \tabularnewline
\hline 
4$^{\circ}$ &  &  &  &  &  &  &  &  &  & \tabularnewline
\hline 
5$^{\circ}$ &  &  &  &  &  &  &  &  &  & \tabularnewline
\hline 
6$^{\circ}$ &  &  &  &  &  &  &  &  &  & \tabularnewline
\hline 
7$^{\circ}$ &  &  &  &  &  &  &  &  &  & \tabularnewline
\noalign{\hrule height 1.5pt}
\end{tabular}
\vfill

\textbf{MATRÍCULA:}

\vspace{0.5cm}

\textbf{NOME:}

\vspace{0.5cm}

\textbf{TURMA:}
\end{document}